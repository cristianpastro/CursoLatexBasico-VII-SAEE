\documentclass[12pt, twocolumn]{article}
\usepackage[utf8]{inputenc}
\usepackage[portuguese]{babel}

\title{Um texto genérico em \LaTeX}
\author{Engenheiro Eletricista da Silva}
\date{\today}

\begin{document}

    \tableofcontents
    \maketitle
    
    Este documento tem por objetivo o treinamento de \ \LaTeX.
    
    \section{Conceitos iniciais}
        
    \textbf{Engenharia elétrica} é o ramo da engenharia que trabalha com os estudos e aplicações de:
    \begin{itemize}
        \item eletricidade;
        \item eletromagnetismo;
        \item eletrônica.
    \end{itemize}
    
    Algumas disciplinas deste curso são: 
    \begin{enumerate}
        \item Geração De Energia Elétrica;
        \item Sistemas de Controle;
        \item Instalações Elétricas;
        \item Máquinas Elétricas.
    \end{enumerate}
    
    \subsection{Negrito e Itálico}
    
    Um conceito moderno para os engenheiros eletricistas é o de \textit{smart grids}. \textit{smart grid} refere-se a um sistema de energia elétrica que se utiliza da tecnologia da informação para fazer com que o sistema seja mais eficiente.\\
    
    Nota-se que \textbf{Rede elétrica }é uma rede interligada para entrega da eletricidade dos fornecedores aos consumidores.
    
    \section{Lei de Ohm}
    
    Todo Engenheiro Eletricista conhece a famosa lei de \textit{Ohm}, dada pela Equação \ref{eq.ohm}.
    
    \begin{equation}
        \label{eq.ohm}
        V = RI.
    \end{equation}
    
    E como vale a Equação \ref{eq.ohm}, sabe-se também que $I = \frac{V}{R}$.
    
    \section{O \LaTeX \ e os engenheiros}
    
    Assim como outros estudantes da área de exatas, engenheiros eletricistas usam bastante o \ \LaTeX, no qual o preâmbulo inicia com:
    \begin{verbatim}
        \documentclass[12pt]{article}
    \end{verbatim}
    
    
    \section{Fórmula para resolução de Equações Quadráticas}
    
    A fórmula de Bháskara é dada por \ref{eq.bas}:
    \begin{equation}
        \label{eq.bas}
        x_{1,2} = \frac{-b \pm \sqrt{\Delta}}{2a},
    \end{equation}
    em que:
    $$
        \Delta = b^2-4ac.
    $$
    
    \subsection{Matrizes}
    
    A matriz de marização é dada pela Equação \ref{eq.mat}.
    \begin{equation}
        \label{eq.mat}
        M = \left[ 
            \begin{array}{ccccc}
            e        &    x_a    &    \alpha &   10  &  \pi\\
            \delta   &    0      &    \beta  &   10  &  \pi \\
            \sigma   &    e^\pi  &    \Delta &   10   & \pi  \\
            3        &    2      &    \omega &   10   & \pi  \\
            \end{array}
        \right]
        \times
        \left[ 
            \begin{array}{c}
            \theta_{\alpha}  \\
            \theta_{\beta}   \\
            \theta_{\gamma}   \\
            \theta_{\omega}   \\
            \theta_{\phi}   \\
            \end{array}
        \right]
    \end{equation}
    
    em que $det(M) = 2^{x^2-3\beta} \ \%$\\
    
    ``texto entre aspas duplas''.\\

     `texto entre aspas simples'.
     
    Dica$^1$ para a equação abixo: usa-se o mesmo conceito da matriz.\\
    
     Dica$^2$ para a equação abixo: você pode escrever um texto literal dentro de ambiente matemático com o comando:
     \begin{verbatim}
         \mbox {texto qualquer}
     \end{verbatim}
    
    
	$$
	|x| =  \left\{
	\begin{array}{lll}
    	x     &   \mbox{se } \leq x ;\\ \\
    	-x    & 	 \mbox{caso contrário}.
	\end{array}
	\right.
	$$

     
    








    
\end{document}